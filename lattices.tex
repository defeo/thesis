In Chapter~\ref{chap:isomorphism}, we have seen algorithms to compute
isomorphisms, or inclusions, between pairs of finite fields. Now, we want to
integrate these algorithms in a larger global system with potentially as many
finite fields as we want.
\minitoc

\begin{figure}%[h]
  \centering

    \tikzset{
        dotstyle/.style={circle, inner sep = 1.2pt, outer sep = 4pt, fill =
        gray},
        edgetower/.style={thick},
        edgecomp/.style={thick, lightgray}
          }
          \begin{tikzpicture}[scale=.9]
    \coordinate (T2) at (-2, 0.5);
    \node (Fp) at (0, 0) {$\mathbb{F}_p$};
    \node (Fp2) at ($(Fp) + (T2)$) {$\mathbb{F}_{p^2}$};
    \node (Fp4) at ($(Fp2) + (T2)$) {$\mathbb{F}_{p^4}$};
    \node (Fp2l) at ($(Fp4) + (T2)$) {};% {$\FF_p^{(2)}$};
    % ---------------------
    \coordinate (T3) at (-0.7, 2);
    \node (Fp3) at ($(Fp) + (T3)$) {$\mathbb{F}_{p^3}$};
    \node (Fp9) at ($(Fp3) + (T3)$) {$\mathbb{F}_{p^9}$};
    \node (Fp3l) at ($(Fp9) + (T3)$) {};% {$\FF_p^{(3)}$};
    % ---------------------
    \coordinate (T5) at (0.7, 2);
    \node (Fp5) at ($(Fp) + (T5)$) {$\mathbb{F}_{p^5}$};
    \node (Fp25) at ($(Fp5) + (T5)$) {$\mathbb{F}_{p^{25}}$};
    \node (Fp5l) at ($(Fp25) + (T5)$) {};% {$\FF_p^{(5)}$};
    % ---------------------
    \coordinate (Tl) at (2, .5);
    \node (Fpl) at ($(Fp) + (Tl)$) {$\mathbb{F}_{p^\ell}$};
    \node (Fpl2) at ($(Fpl) + (Tl)$) {$\mathbb{F}_{p^{\ell^2}}$};
    \node (Fpll) at ($(Fpl2) + (Tl)$) {};% {$\FF_p^{(\ell)}$};
    % ---------------------
    \node[dotstyle] (dot1) at ($(Fp2) + (Fp3) - (Fp)$) {};
    \node[dotstyle] (dot2) at ($(Fp4) + (dot1) - (Fp2)$) {};
    \node[dotstyle] (dot3) at ($(Fp2) + (Fp5) - (Fp)$) {};
    \node[dotstyle] (dot4) at ($(Fp3) + (Fp5) - (Fp)$) {};
    \node[dotstyle] (dot5) at ($(Fp3) + (Fpl) - (Fp)$) {};
    \node[dotstyle] (dot6) at ($(Fp5) + (Fpl) - (Fp)$) {};
    \node[dotstyle] (dot7) at ($(Fpl2) + (dot6) - (Fpl)$) {};
    % ---------------------
    \draw
    (Fp)
    edge[edgetower] (Fp2)
    edge[edgetower] (Fp3)
    edge[edgetower] (Fp5)
    edge[edgetower] (Fpl)
    (Fp2)
    edge[edgetower] (Fp4)
    edge[edgecomp] (dot1)
    (Fp4)
    edge[edgetower, dotted] (Fp2l)
    edge[edgecomp] (dot2)
    (dot1)
    edge[edgecomp] (dot2)
    (Fp3)
    edge[edgetower] (Fp9)
    edge[edgecomp] (dot1)
    edge[edgecomp] (dot4)
    (Fp9)
    edge[edgetower, dotted] (Fp3l)
    (Fp5)
    edge[edgetower] (Fp25)
    edge[edgecomp] (dot4)
    edge[edgecomp] (dot6)
    (Fp25)
    edge[edgetower, dotted] (Fp5l)
    (Fpl)
    edge[edgetower] (Fpl2)
    edge[edgecomp] (dot6)
    (Fpl2)
    edge[edgetower, dotted] (Fpll)
    edge[edgecomp] (dot7)
    (dot3)
    edge[edgecomp] (Fp2)
    edge[edgecomp] (Fp5)
    (dot5)
    edge[edgecomp] (Fp3)
    edge[edgecomp] (Fpl)
    (dot6)
    edge[edgecomp] (dot7);
  \end{tikzpicture}
  \caption{Extensions de $\mathbb{F}_p$.}
  \label{fig:alg-closure}
\end{figure}

\clearpage
\section{The compatibility problem}

Now that we know how to go from one finite field $\mathbb{F}_{p^a}$ to another
$\mathbb{F}_{p^b}$, with $p\in\mathbb{N}$ a prime number and
\[
  a\,|\,b
\]
two integers, we would like to be able to manage more than $2$ finite fields
simultaneously. In other words, given a family
\[
  \F=\left\{ \mathbb{F}_{p^{a}}\,|\,a\in E \right\}
\]
for $E$ a subset of $\mathbb{N}\setminus\left\{ 0 \right\}$,
we want to be able to compute an embedding
\[
  \mathbb{F}_{p^a}\emb\mathbb{F}_{p^b}
\]
each time that we have $\mathbb{F}_{p^a}, \mathbb{F}_{p^b}\in\F$ with $a$
dividing $b$. On a pratical point of view, we want to build a computer algebra
system where the users can embed the field they are working with in a bigger
finite field, or conversely if an element is known to belong to a smaller field
than the ambient field, project it to a smaller field. In cryptology and coding
theory, finite fields are ubiquitus, and some algorithms require frequent change
of field. For example, in the
quasi-polynomial algorithm for discrete logarithm in small characteristic by
Granger, Kleinjung and Zumbrägel~\cite{GKZ14}, we have to work with a tower of
finite field extensions, and thus a computer algebra system automatically
dealing with the changes would be very convenient in order to implement their
algorithm. On a theoretical point of view, this leads to the question of the
arithmetic in the algebraic closure of some finite field $\mathbb{F}_p$
\[
  \bar{\mathbb{F}}_p=\bigcup_{j\in\mathbb{N}\setminus\left\{ 0
  \right\}}\mathbb{F}_{p^j}
\]
and its representation on a computer, a question that was for example
investigated in~\cite{DDS14}. 

If we \emph{just} want to be able to compute embeddings between two finite
fields in $\F$ when it makes sense, we can just use one of the algorithms presented
in Chapter~\ref{chap:isomorphism} each time we need it. In fact, we want to
build a data structure $\Lambda$ to represent all the extensions of
$\mathbb{F}_p$ that we need, and additionnal sub-goals might be desirable.
\begin{description}
\item[\emph{Effective embeddings:}] for any pair of extensions
  $k\subset K$ in $\Lambda$, there exists an efficiently computable
  embedding $\phi:k\to K$, and algorithms to evaluate $\phi$ on $k$,
  and the section $\phi^{-1}$ on $K$.
\item[\emph{Compatibility:}] the embeddings are \emph{compatible},
  \ie for any triple $k\subset K\subset L$ in $\Lambda$, and
  embeddings $\phi:k\to K$, $\psi:K\to L$, $\chi:k\to L$ such as shown in
  Figure~\ref{fig:compatibility}, one has
  $\chi=\psi\circ\phi$.
  \begin{figure}[h]
    \centering
    \begin{tikzpicture}
      \node (E) at (0, 0) {$k$};
      \node (F) at (1.5, 1) {$K$};
      \node (G) at (0.5, 2) {$L$};

      \draw[arrow] (E) -- (F);
      \draw[arrow] (E) -- (G);
      \draw[arrow] (F) -- (G);

      \node (f12) at (1, 0.25)
      {$\phi$};
      \node (f13) at (-0.1, 1)
      {$\chi$};
      \node (f23) at (1.4, 1.65)
      {$\psi$};
    \end{tikzpicture}

  \caption{Embeddings between finite fields $k\subset K\subset L$.}
  \label{fig:compatibility} 
  \end{figure}
\item[\emph{Incrementality:}] the data associated with an extension
  (\eg its irreducible polynomial, change-of-basis matrices, \dots)
  must be computable efficiently and \emph{incrementally}, \ie 
  adding a new field extension to $\Lambda$ does not require
  recomputing data for all extensions already in $\Lambda$.
\item[\emph{Uniqueness:}] any extension of $\mathbb{F}_{p}$ is determined by an
  irreducible polynomial whose definition only depends on the
  characteristic $p$ and the degree of the extension.
\item[\emph{Generality:}] extensions of $\mathbb{F}_{p}$ can be represented by
  arbitrary irreducible polynomials.
\end{description}
We cannot fulfill all these conditions simultaneously, as
uniqueness and generality are in conflict with each other. One might prefer to
look for uniqueness of generality, depending on the situation, but it would
always be better to gave effective embeddings, compatibility and incrementality.
The two constructions presented in Sections~\ref{sec:conway}
and~\ref{sec:bosma-canon-steel} are both solutions to achieve
\emph{compatibility}.

\section{Conway polynomials}
\label{sec:conway}
\section{Bosma-Canon-Steel framework}
\label{sec:bosma-canon-steel}
