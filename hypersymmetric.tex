In Chapter~\ref{chap:bilinear}, we have seen the notions of bilinear complexity
and symmetric bilinear complexity. We now investigate even stronger
notions of symmetry, allowing to have very short representations of a bilinear
map.

\minitoc

% TODO
% ====
%
% Find a nice picture to put here to illustrate something in link with the
% chapter.

\clearpage
\section{Symmetric and hypersymmetric fomulas}
% Table of content
% ================
%
% - Recall of the definition of symmetric
% - Existence and lemma for the symmetric case
% - non degenerate bilinear form, link with the trace, but not only
% - Link between symmetric and hypersymmetric in smaller dimension
% - Galois invariance
% - Comment for the case of the particular algebras we study, what is known and
%   what is not
%
% Comment
% =======
%
% Comment about the trisymmetric formulas, it is true that F_4/F_2 can be
% represented by a trisymmetric formula but it is not the case for F_8/F_2. We
% should check the lemma saying something on the existence of the trisymmetric
% decomposition, but it is probably just simpler.

Let $\K$ be a finite field, $V_1$, $V_2$ and $W$ three finite-dimensional $\K$-vector
space and
\[
  \Phi:V_1\times V_2\to W
\]
a bilinear map. Recall Definition~\ref{defi:bilinear-formula}:
\[
  \Phi(x, y) = \sum_{j=1}^t\varphi_j(x)\psi_j(y)w_j,
\]
where for all $1\leq j\leq t$, $\varphi_j\in V_1^\vee$ and $\phi_j\in V_2^\vee$ are linear forms and
$w_j\in W$ is a vector, is called a \emph{bilinear formula} of length $t$. If
the spaces $V_1$ and $V_2$ are equal and if the bilinear map $\Phi$ is
symmetric, \ie if for all $x, y\in V$
\[
  \Phi(x, y) = \Phi(y, x),
\]
we can investigate the existence of formulas satisfying the same condition of
symmetry, \ie formulas where for all $1\leq j\leq t$, $\varphi_j=\psi_j$,
resulting in \emph{symmetric} bilinear form:
\[
  \Phi(x, y) = \sum_{j=1}^t\varphi_j(x)\varphi_j(y)w_j.
\]
In fact, we can define other interesting types of symmetries, but it is useful
to first generalize the notions that we saw in Chapter~\ref{chap:bilinear} to
higher dimensions.
\subsection{Generalization to multilinear maps}
The definitions of bilinear formula and bilinear complexity are not bound to the
bilinear case and can be generalized to arbitrary dimension. These general
definitions will be used in Section~\ref{subsec:trisym} to define
\emph{hypersymmetric} complexity.
\begin{defi}[Multilinear formula]
Let $V_1, V_2, \dots, V_s$ and $W$ be $s+1$ finite-dimensional $\K$-vector
spaces and
\[
  \Phi:V_1\times V_2\times\dots\times V_s\to W
\]
an $s$-linear map. A \emph{multilinear formula}, or \emph{multilinear
decomposition}, or \emph{multilinear algorithm} of length $t$ for $\Phi$ is a
collection of $s\times t$ linear forms $\varphi_1^{(1)}, \varphi_2^{(1)}, \dots,
\varphi_t^{(1)}\in V_1^\vee$ up to $\varphi_1^{(s)}, \varphi_2^{(s)}, \dots,
\varphi_t^{(s)}\in V_s^{\vee}$ and $t$ vectors $w_1, \dots, w_t$, such that for all $x_1\in V_1, \dots, x_s\in
V_s$, we have
\[
  \Phi(x_1, \dots, x_s) =
  \sum_{j=1}^t\varphi_j^{(1)}(x_1)\dots\varphi_j^{(s)}(x_s)w_j.
\]
\end{defi}
\begin{defi}[Multilinear complexity]
Let $V_1, V_2, \dots, V_s$ and $W$ be $s+1$ finite-dimensional $\K$-vector
spaces and
\[
  \Phi:V_1\times V_2\times\dots\times V_s\to W
\]
an $s$-linear map. The \emph{multilinear complexity} $\mu(\Phi)$ of $\Phi$ is the
minimal length $t$ of a multilinear formula for $\Phi$.
\end{defi}
As in the case of bilinear complexity, the multilinear complexity $\mu(\Phi)$ of a
multilinear map $\Phi$ can also be defined as the rank of the tensor in 
\[
  V_1^\vee\otimes\dots\otimes V_s^\vee\otimes W
\]
corresponding to $\Phi$, see Example~\ref{ex:bilinear-complexity} for an
illustration of this correspondence in the bilinear case. In the case where
\[
  V_1 = V_2 = \dots = V_s,
\]
symmetric formulas and
symmetric complexity can also be generalized when $\Phi$ is a \emph{symmetric}
multilinear map, \ie when for all permutation $\sigma\in\mathfrak S_s$ and for
all vectors $x_1, \dots, x_s\in V$, we have
\[
  \Phi(x_1, \dots, x_s) = \Phi(\sigma(x_1), \dots, \sigma(x_s)).
\]
\begin{defi}[Symmetric multilinear formula]
Let $V$ and $W$ be two finite-dimensional $\K$-vector
spaces and
\[
  \Phi:\underset{\textrm{$s$ times}}{\underbrace{V\times\dots\times V}}\to W
\]
a symmetric $s$-linear map. A \emph{symmetric multilinear formula}, or
\emph{symmetric multilinear
decomposition}, or \emph{symmetric multilinear algorithm} of length $t$ for $\Phi$ is a
collection of $t$ linear forms $\varphi_1, \varphi_2, \dots,
\varphi_t\in V^\vee$ and $t$ vectors $w_1, \dots, w_t$, such that for all $x_1, \dots, x_s\in
V$, we have
\[
  \Phi(x_1, \dots, x_s) =
  \sum_{j=1}^t\varphi_j(x_1)\dots\varphi_j(x_s)w_j.
\]
\end{defi}
\begin{defi}[Symmetric multilinear complexity]
Let $V$ and $W$ be two finite-dimensional $\K$-vector
spaces and
\[
  \Phi:\underset{\textrm{$s$ times}}{\underbrace{V\times\dots\times V}}\to W
\]
a symmetric $s$-linear map. The \emph{symmetric multilinear complexity} $\musym(\Phi)$ of $\Phi$ is the
minimal length $t$ of a symmetric multilinear formula for $\Phi$. If no such
formula exists, we set
\[
  \musym(\Phi) = \infty.
\]
\end{defi}
Contrary to the bilinear case, some symmetric multilinear maps do not admit a symmetric
decomposition, but the problem of whether a symmetric multilinear map admits
a symmetric multilinear formula is well understood and follows from
Theorem~\ref{thm:symmetric-formula}.
\begin{thm}[{\cite[Thm.~A.7]{Randriam15}}]\label{th:criterion}
\label{thm:symmetric-formula}
Let $\Phi:V^s\to W$ be a $s$-linear map between finite dimensional vector spaces over $\mathbb{F}_q$.
Then $\Phi$ admits a symmetric decomposition if and only if $\Phi$ is \emph{Frobenius-symmetric},
\ie if and only if it is symmetric and one of the following two conditions holds:
\begin{itemize}
\item $s\leq q$
\item $s\geq q+1$ and for all $u,v,z_1,\dots,z_{s-q-1}$ in $V$,
\[
\Phi(\underset{\textrm{$q$ times}}{\underbrace{u,\dots,u}},v,z_1,\dots,z_{s-q-1})=\Phi(u,\underset{\textrm{$q$ times}}{\underbrace{v,\dots,v}},z_1,\dots,z_{s-q-1}).
\]
\end{itemize}
\end{thm}

\subsection{Trisymmetric and hypersymmetric complexity}
\label{subsec:trisym}

Under even stricter conditions, we can study the existence of even more
symmetric formulas. These formulas allow to describe a multilinear map with
fewer elements, and thus give a compact definition of the map. Since the
symmetry conditions are stronger there are fewer such formulas, and as a
consequence the search space is smaller. Thus, we expect algorithms to be
faster, as was the case when using Barbulescu \etal algorithm
(Algorithm~\ref{algo:BDEZ}) to find symmetric formulas. Let us define those
``stricter conditions''. Let
\[
  \Phi:V^s\to V
\]
be an $s$-linear symmetric map, \ie we additionnaly ask that $W=V$. We also
assume that $V$ has a non-degenerate symmetric bilinear form, that we write as a
scalar product
\[
 \begin{array}{ccc}
 V\times V &\to&\K\\
 (v,w)&\mapsto&\ps{v}{w}.
 \end{array}
\]
In that case, we know that the vector space $V$ is isomorphic to its dual space
$V^\vee$:
\[
  V\cong V^\vee,
\]
\ie for each linear form $\varphi\in V^\vee$, there exist a unique vector $a\in
V$ such that for all $x\in V$, we have
\[
  \varphi(x) = \ps{a}{x}.
\]
Under these conditions, we can now write a symmetric formula for $\Phi$ as
\[
  \Phi(x, y) = \sum_{j=1}^t\ps{a_i}{x}\ps{a_i}{y}w_j
\]
where for all $1\leq j\leq t$, $a_j\in V$ is a vector of $V$. As a consequence,
we can also describe a symmetric formula for $\Phi$ as the data of vectors
$(a_j)_{1\leq j\leq t}$ and $(w_j)_{1\leq j\leq t}$. In order to have an even
more compact description of $\Phi$, one can ask for the vectors $w_j$ to be
proportional to $a_i$, leading to the definition of hypersymmetric
formula.
% Note
% ====
%
% But is that really natural? Wouldn't it be better to say that we would like
% the a_i and the b_i to be equal? 
%
% TODO
% ====
%
% Maybe change this to include the case a_i = b_i and discuss it a little, or
% maybe not, we'll see.
%
% Remark
% ======
%
% If this is presented, it could be nice to have the tensor point of view, with
% a decomposition with the same elements three times being then very clear.

\begin{defi}[Hypersymmetric formula]
Let $V$ a finite-dimensional $\K$-vector
spaces equipped with a saclar product and
\[
  \Phi:\underset{\textrm{$s$ times}}{\underbrace{V\times\dots\times V}}\to W
\]
a symmetric $s$-linear map. A \emph{hypersymmetric formula}, or
\emph{hypersymmetric decomposition}, or \emph{hypersymmetric algorithm} of length $t$ for $\Phi$ is a
collection of $t$ vectors $a_1, \dots, a_t\in V$ and $t$ scalars $\lambda_1,
\dots, \lambda_t\in\K$, such that for all $x_1, \dots, x_s\in
V$, we have
\[
  \Phi(x_1, \dots, x_s) =
  \sum_{j=1}^t\lambda_j\ps{a_j}{x_1}\dots\ps{a_j}{x_s}a_j.
\]
\end{defi}
\begin{defi}[Hypersymmetric complexity]
Let $V$ be a finite-dimensional $\K$-vector spaces equipped with a saclar product
and
\[
  \Phi:\underset{\textrm{$s$ times}}{\underbrace{V\times\dots\times V}}\to W
\]
a symmetric $s$-linear map. The \emph{hypersymmetric complexity} $\muhyp(\Phi)$ of $\Phi$ is the
minimal length $t$ of a hypersymmetric formula for $\Phi$. If no such
formula exists, we set
\[
  \muhyp(\Phi) = \infty.
\]
\end{defi}
\begin{ex}
  \label{ex:trisymmetric-formula}
  We take the same case as in Example~\ref{ex:bilinear-complexity}, but viewed
  a bit differently. Let $\K=\mathbb{F}_2$ and 
  \[
    V=\mathbb{F}_4\cong\mathbb{F}_2[T]/(T^2+T+1)\cong\mathbb{F}_2(\zeta)
  \]
  seen as a $\mathbb{F}_2$-vector space of dimension $2$ using the base $(1,
  \zeta)$. The $\mathbb{F}_2$-bilinear map $\Phi$ that we
  consider is the product in $\mathbb{F}_4$:
  \[
 \begin{array}{cccc}
   \Phi: & \mathbb{F}_4\times \mathbb{F}_4 &\to&\mathbb{F}_4\\
 &(x,y)&\mapsto&xy.
 \end{array}
  \]
  We also consider the non-degenerate symmetric bilinear form
\[
 \begin{array}{ccc}
   \mathbb{F}_4\times \mathbb{F}_4 &\to&\mathbb{F}_2\\
 (v,w)&\mapsto&\tr(vw),
 \end{array}
\]
where $\tr$ is the trace of the field extension $\mathbb{F}_4/\mathbb{F}_2$,
and we write
\[
  \tr(xy) = \ps{x}{y}.
\]
If $x = x_0 + x_1\zeta\in\mathbb{F}_4$ is an element in the extension field, we have $\tr(x) = x_1$, and if $y = y_0
+ y_1\zeta\in\mathbb{F}_4$ is another element, then their product is
\[
  xy = x_0y_0 + x_1y_1 + (x_0y_1 + x_1y_0 + x_1y_1)\zeta.
\]
We also see that
\[
\left\{ 
  \begin{array}{lll}
    \ps{1}{x}\ps{1}{y} &=& x_1y_1 \\
    \ps{1+\zeta}{x}\ps{1+\zeta}{y} &=& x_0y_0 \\
    \ps{\zeta}{x}\ps{\zeta}{y} &=& (x_0+x_1)(y_0+y_1)
  \end{array}
\right.
\]
and thus we have
\[
  xy =
  \ps{1}{x}\ps{1}{y}\cdot1+\ps{1+\zeta}{x}\ps{1+\zeta}{y}\cdot(1+\zeta)+\ps{\zeta}{x}\ps{\zeta}{y}\cdot\zeta.
\]
This is an hypersymmetric formula of length $3$, and we can prove that there are
no formulas of length $2$, so we have
\[
  \muhyp(\Phi) = 3.
\]
This is in fact the very same formula as in
Example~\ref{ex:bilinear-complexity}.
\end{ex}
In order to investigate the existence of hypersymmetric decompositions, we
remark that there is a natural link between hypersymmetric decompositions of 
the $s$-linear map
\[
  \Phi:V^s\to V
\]
and symmetric decompositions of the $(s+1)$-linear form $\widetilde\Phi$ defined by
\[
  \begin{array}{llll}
    \widetilde\Phi:&V^{s+1}&\to&\K\\
    &(x_1, \dots, x_{s+1})&\mapsto&\ps{\Phi(x_1, \dots, x_s)}{x_{s+1}}
  \end{array}
\]
in the sense of~Lemma~\ref{lm:link-hyp-sym}. Moreover, we say that the
$s$-linear map $\Phi$ is \emph{hypersymmetric} if the associated $(s+1)$-linear
form $\widetilde\Phi$ is symmetric.
\begin{lm}
  \label{lm:link-hyp-sym}
  Let $V$ a $\K$-vector space and 
  \[
    \Phi:V^s\to V
  \]
  a symmetric $s$-linear map. 
Elements $(a_j)_{1\leq j\leq t}$ in $V$ and scalars $(\lambda_j)_{1\leq j\leq t}$ in $\K$ define a hypersymmetric formula for the $s$-linear map $\Phi$,
\[
\Phi(x_1,\dots,x_s)=\sum_{j=1}^{t}\lambda_j\ps{a_j}{x_1}\cdots\ps{a_j}{x_t}a_j,
\]
if and only if they define a symmetric formula for the $(s+1)$-linear form $\widetilde{\Phi}$,
\[
\widetilde{\Phi}(x_1,\dots,x_s,x_{s+1})=\sum_{j=1}^{t}\lambda_i\ps{a_j}{x_1}\cdots\ps{a_j}{x_t}\ps{a_j}{x_{s+1}}.
\]

Thus, $\Phi$ admits a hypersymmetric formula if and only if $\widetilde{\Phi}$ is Frobenius-symmetric (in the sense of Theorem~\ref{thm:symmetric-formula}),
and we have
\[
\muhyp(\Phi)=\musym\left(\widetilde{\Phi}\right).
\]

In particular, if $q\geq s+1$, then any hypersymmetric $s$-linear map over $\mathbb{F}_q$ admits a hypersymmetric formula.
\end{lm}
\begin{proof}
  Assume that $\Phi$ admits a hypersymmetric decomposition, such that for all
  $x_1, \dots, x_s\in V$, we have
  \[
    \Phi(x_1,\dots,x_s)=\sum_{j=1}^{t}\lambda_i\ps{a_j}{x_1}\cdots\ps{a_j}{x_t}a_j,
  \]
  then, by taking the scalar product with any $x_{s+1}$, we obtain
\[
  \ps{\Phi(x_1, \dots,
  x_s)}{x_{s+1}}=\widetilde{\Phi}(x_1,\dots,x_s,x_{s+1})=\sum_{j=1}^{t}\lambda_i\ps{a_j}{x_1}\cdots\ps{a_j}{x_t}\ps{a_j}{x_{s+1}},
\]
which defines a symmetric decomposition for $\widetilde\Phi$. In the other
direction, assume that $\widetilde\Phi$ admits a symmetric decomposition, such
that for all $x_1, \dots, x_{s+1}\in V$, we have
\[
\widetilde{\Phi}(x_1,\dots,x_s,x_{s+1})=\sum_{j=1}^{t}\lambda_i\ps{a_j}{x_1}\cdots\ps{a_j}{x_t}\ps{a_j}{x_{s+1}}.
\]
It can also be written as
\[
  \ps{\Phi(x_1, \dots,
  x_s)}{x_{s+1}}=\ps{\sum_{j=1}^t\lambda_j\ps{a_j}{x_1}\cdots\ps{a_j}{x_s}a_j}{x_{s+1}},
\]
so that we have
\[
  \ps{\Phi(x_1, \dots,
  x_s)-\sum_{j=1}^t\lambda_j\ps{a_j}{x_1}\cdots\ps{a_j}{x_s}}{x_{s+1}}=0.
\]
Since the scalar product $\ps{\cdot}{\cdot}$ is non-degenerate, it means that
  \[
    \Phi(x_1,\dots,x_s)=\sum_{j=1}^{t}\lambda_i\ps{a_j}{x_1}\cdots\ps{a_j}{x_t}a_j.
  \]
  Hence $\Phi$ admits a hypersymmetric decomposition. The other assertions
  follow.
\end{proof}
The most important case is arguably the bilinear case, where $s=2$, because it
was thoroughly studied. For that reason, we sometimes replace the word
hypersymmetric by \emph{trisymmetric} in that particular case, because of the
form of the formulas
\[
  \Phi(x, y) = \sum_{j=1}^t\lambda_j\ps{a_j}{x}\ps{a_j}{y}a_j
\]
that includes the same element $a_j$ three times, and we write $\mutri(\Phi)$
instead of $\muhyp(\Phi)$. Lemma~\ref{lm:link-hyp-sym} states that, if $q\geq3$, a
trisymmetric map $\Phi$ always admits a trisymmetric decomposition.

\subsection{Galois invariance}

An other type of interesting decompositions is Galois invariant decompositions.
It is motivated by the study of group actions on the set of decompositions, that
can sometimes be used to cut branches in the search tree of the algorithms.
% TODO
% ====
%
% Link that with BDEZ stab, the version using automorphisms, and Covanov.
Let 
\[
 \begin{array}{cccc}
   \sigma: & V\times V &\to&V\\
 &x&\mapsto&x^\sigma
 \end{array}
\]
be a $\K$-automorphism of $V$ that respects the scalar product, \ie for all $x,
y\in V$, we have
\[
  \ps{x^\sigma}{y^\sigma} = \ps{x}{y}.
\]
Then, if $\sigma$ is also compatible with some studied multilinear map $\Phi$, it induces
an action on the set of decomposition, as explained in
Lemma~\ref{lm:action-sym}.
\begin{lm}
  \label{lm:action-sym}
  Let $V$ a finite-dimensional $\K$-vector space and
  \[
    \Phi:V^s\to V
  \]
  a symmetric $s$-linear map that is compatible with $\sigma$, \ie for all $x_1,
  \dots, x_s$ in $V$, we have
  \[
    \Phi(x_1^\sigma, \dots, x_s^\sigma) = \Phi(x_1, \dots, x_s)^\sigma.
  \]
  If $(a_j)_{1\leq j \leq t}$ and $(b_j)_{1\leq j \leq t}$ define a symmetric
  formula for $\Phi$
  \[
    \Phi(x_1, \dots, x_s) = \sum_{j=1}^t\ps{a_j}{x_1}\dots\ps{a_j}{x_s}b_j,
  \]
  then $(a_j^\sigma)_{1\leq j\leq t}$ and $(b_{j}^\sigma)_{1\leq j\leq t})$ also
  define a symmetric formula for $\Phi$
  \[
    \Phi(x_1, \dots, x_s) =
    \sum_{j=1}^t\ps{a_j^\sigma}{x_1}\dots\ps{a_j^\sigma}{x_s}b_j^\sigma.
  \]
\end{lm}
\begin{proof}
 Assume that we have a symmetric decomposition for $\Phi$, with the same
 notations as in the Lemma. First, notice that for every $x,y\in V$, we have
 \[
   \ps{x^\sigma}{y} = \langle{x},{y^{\sigma^{-1}}}\rangle.
 \]
 Then, it follows that
 \begin{align*}
   \Phi(x_1, \dots, x_s) &= \Phi(x_1^{\sigma^{-1}}, \dots,
   x_{s}^{\sigma^{-1}})^\sigma\\
   &=
   (\sum_{j=1}^t\langle{a_j},{x_1^{\sigma^{-1}}}\rangle\dots\langle{a_j},{x_s^{\sigma^{-1}}}\rangle
   b_j)^\sigma\\
   &= \sum_{j=1}^t\ps{a_j^{\sigma}}{x_1}\dots\ps{a_j^\sigma}{x_s}b_j^\sigma.
 \end{align*}
 Thus we have a new symmetric formula for $\Phi$.
\end{proof}
When this action, does not change the formula, we then say that it is
$\sigma$-invariant.
\begin{defi}[$\sigma$-invariance]
  Let $(a_j)_{1\leq j\leq t}$ and
$(b_j)_{1\leq j\leq t}$ define a symmetric formula for $\Phi$
  \[
    \Phi(x_1, \dots, x_s) = \sum_{j=1}^t\ps{a_j}{x_1}\dots\ps{a_j}{x_s}b_j.
  \]
We say that this
formula is $\sigma$-invariant if it is the same as the formula defined by
$(a_j^\sigma)_{1\leq j\leq t}$ and $(b_j^\sigma)_{1\leq j\leq t}$,
\ie if there is a permutation $\pi\in\mathfrak S_t$ of $\{1,\dots,t\}$ such that
$(a_j^\sigma,b_j^\sigma)=(a_{\pi(j)},b_{\pi(j)})$ for all $j$. This also applies
to hypersymmetric formulas, setting $b_j=\lambda_j a_j$. 
\end{defi}
\begin{ex}
  In fact, the trisymmetric formula seen in
  Example~\ref{ex:trisymmetric-formula} was already $\sigma$-invariant. Recall
  that we work in $\mathbb{F}_4$, defined by
  \[
   \mathbb{F}_4\cong\mathbb{F}_2[T]/(T^2+T+1)\cong\mathbb{F}_2(\zeta),
  \]
  and we take
\[
  \tr(xy) = \ps{x}{y}.
\]
The $\mathbb{F}_2$-automorphism that we consider is the Frobenius automorphism
\[
  \begin{array}{cccc}
    \sigma: & \mathbb{F}_4 & \to & \mathbb{F}_4 \\
    & x & \mapsto & x^2.
  \end{array}
\]
We still have, for all $x, y\in\mathbb{F}_4$
\[
  xy =
  \ps{1}{x}\ps{1}{y}\cdot1+\ps{1+\zeta}{x}\ps{1+\zeta}{y}\cdot(1+\zeta)+\ps{\zeta}{x}\ps{\zeta}{y}\cdot\zeta.
\]
Since
\[
\left\{ 
  \begin{array}{lll}
    1^\sigma = 1 \\
    \zeta^\sigma=\zeta+1\\
    (\zeta+1)^\sigma=\zeta
  \end{array}
\right.
\]
we wee that the formula is $\sigma$-invariant.
\end{ex}

\subsection{Multiplication formulas in algebras}

In the previous pages, we have defined (hyper)symmetric formulas for any
multilinear map defined over some finite-dimensional $\K$-vector space.
Nevertheless, as seen in the examples, we are often interested in special
instances of multilinear maps. In fact, the map that we have in mind is almost
always the $s$-variable product of some $\K$-algebra $\A$, with $s=2$. There are
two types of algebras we are particularly interested in:
\begin{itemize}
  \item the finite field extensions $\A=\mathbb{F}_{q^m}$, in which we take the
    trace bilinear form
    \[
      \ps{x}{y}=\tr(xy)
    \]
    for our scalar product, and the Frobenius automorphism
    \[
  \begin{array}{cccc}
    \sigma: & \mathbb{F}_{q^m} & \to & \mathbb{F}_{q^m} \\
    & x & \mapsto & x^q.
  \end{array}
\]
for our $\K$-automorphism ;
\item algebras of truncated polynomials $\A = \mathbb{F}_q[T]/(T^m)$. In this
  case, we let 
  \[
  \begin{array}{cccc}
    \tau: & \A & \to & \K \\
    & \sum_{j=0}^{m-1}x_j T^j & \mapsto & x_0
  \end{array}
\]
and
\[
  \ps{x}{y} = \tau(xy).
\]
Indeed, if $x=\sum_{j=0}^{m-1}T^j$ and $y=\sum_{j=0}^{m-1}y_jT^j$, we have
\[
  \tau(x, y) = x_0y_{m-1} + x_1y_{m-2} + \dots + x_{m-1}y_0,
\]
thus $\tau$ is a non-degenerate bilinear form.
\end{itemize}
We denote by $\mutri_q(k)$ the trisymmetric bilinear complexity of the
$2$-variable product in $\mathbb{F}_{q^k}$ and by $\hmutri_q(k)$ the
trisymmetric bilinear complexity of the $2$-variable product in
$\mathbb{F}_q[T]/(T^k)$.

\section{Algorithmic search in small dimension}

In very small dimension, \eg in Examples~\ref{ex:bilinear-complexity}
and~\ref{ex:trisymmetric-formula} where we are working with $\K$-vector spaces
of dimension $2$, the search of formulas can be done by hand relatively easily
because the length of the formulas is short. Though, even in small dimension, if
the cardinality of $\K$ is large, the number of different formulas can be
big, thus making it hard to list \emph{all} different solutions. For these
reasons, it is highly desirable to \emph{algorithmically} find the formulas.
Let $V$ be a finite dimensional $\K$-vector space and $\Phi:V\times V\to V$ a bilinear map.
When wanting to list all the symmetric formulas of the form
\[
  \Phi(x, y) = \sum_{j=0}^t\varphi_j(x)\varphi_j(y)a_j,
\]
Barbulescu~\etal\!\!\!'s and Covanov's algorithms exhaustively search through
all linear forms $\varphi$, cleverly eliminating useless branches in the search
tree. Nevertheless, their methods exploit the fact that it is possible to freely
choose the vectors $a_j\in V$, independently of the linear forms $\varphi_j\in
V^\vee$. This is
no longer possible when searching for trisymmetric decompositions
\[
  \Phi(x, y) = \sum_{j=0}^t\lambda_j\ps{a_j}{x}\ps{a_j}{y}a_j,
\]
because each
linear form $\varphi\in V^\vee$ is linked with a unique vector $a\in V$ such
that for all $x\in V$
\[
  \varphi(x) = \ps{a}{x},
\]
and the choice of a linear form $\varphi_j$ with $\varphi_j(x)=\ps{a_j}{x}$
imposes the choice of the vector $a_j\in V$. We thus propose an \emph{ad hoc}
algorithm to find trisymmetric formulas, exploiting the vector space structure.

% TODO
% ====
%
% Explain in *details* the algorithms, the different ideas to represent the
% bilinear forms, the margins, the fact that some elements can be forgotten
% without them, the way it is implemented in Julia, etc
%
% Note
% ====
%
% Maybe begin with the general framework/idea, like in the WAIFI paper, and
% continue with the different possibilites for the implementation later

\subsection{General algorithm description}

In all the section, we assume that $\K=\mathbb{F}_q$ is the finite field with
$q$ elements, $V$ is a finite-dimensional $\K$-vector space equipped with a
non-degenerate bilinear form, written as a sclar product $\ps{\cdot}{\cdot}$,
and $\Phi:V\times V\to V$ is a hypersymmetric bilinear map for which we want to
compute trisymmetric decompositions
\[
  \Phi(x, y) = \sum_{j=1}^t\lambda_j\ps{a_j}{x}\ps{a_j}{y}a_j.
\]
Assume that
\[
  V\cong\K^m
\]
is a $\K$-vector space of dimension $m$, for which a basis has been chosen and
allows us to identify $V$ to $\K^m$, and
let $(b_j)_{1\leq j\leq m}$ be the bilinear forms on each coordinates, \ie for
all $x,y\in V$, we have
\[
  \Phi(x, y) = (b_1(x, y), \dots, b_m(x, y)).
\]
We have already seen that the difficulty in the trisymmetric case resides in the
fact that each linear form, or equivalently each symmetric rank $1$ bilinear
form, comes with a given vector in $V$ that dictates its impact on the different
coordinates in a trisymmetric formula. Therefore, a central idea is to
exhaustively search through vectors in $V$ instead of linear forms in $V^\vee$.
This is equivalent since 
\[
  V\cong V^\vee
\]
in this case anyway. Moreover, we search through special sets of vectors that
have easy-to-manage coordinates, \eg well-placed zeros and ones, in order to
control the impact on certain coordinates. Assume $(a_j)_{1\leq j \leq t}$ in $V$ and
$(\lambda_j)_{1\leq j \leq t}$ in $\K$ define a trisymmetric decomposition
\[
  \Phi(x, y) = \sum_{j=1}^t\lambda_j\ps{a_j}{x}\ps{a_j}{y}a_j.
\]
Without loss of generality, we can consider that every element $a_j$ is
``normalized'', \ie its first nonzero coordinate is $1$. Indeed, if we have one
\[
  a_{j_0} = 0
\]
for some $1\leq j_0 \leq t$, then we just remove one term from the formula and
we still have a trisymmetric decomposition, of length $t-1$. Now if for every
$1\leq j\leq t$, 
\[
  a_j\neq0,
\]
we let $x_j$ be the first nonzero coordinate of $a_j$ and we write
\[
  a_j = x_j \widetilde a_j.
\]
We can now write the trisymmetric formula as
\begin{align*}
  \Phi(x, y) &= \sum_{j=1}^t\lambda_j\ps{a_j}{x}\ps{a_j}{y}a_j \\
  &= \sum_{j=1}^t\lambda_j\ps{x_j\widetilde a_j}{x}\ps{x_j\widetilde
  a_j}{y}x_j\widetilde a_j \\
  &= \sum_{j=1}^t\lambda_jx_j^3\ps{\widetilde a_j}{x}\ps{\widetilde
  a_j}{y}\widetilde a_j \\
  &= \sum_{j=1}^t\widetilde \lambda_j\ps{\widetilde a_j}{x}\ps{\widetilde
  a_j}{y}\widetilde a_j,
\end{align*}
where $\widetilde\lambda_j = \lambda_jx_j^3$.
Therefore any trisymmetric formula is equivalent to a trisymmetric formula with
normalized vectors, and thus we only search for formulas with normalized
elements. In other words, for all $1\leq j\leq m$, we let
\[
  \E_j=\left\{ x=(x_1, \dots, x_k)\in V
  \,|\, \forall i\leq j-1,\,x_i=0\text{ and }x_j=1 \right\}
\]
and
\[
  \E = \bigcup_{j=1}^m\E_j,
\]
and we search for elements $a_j$ in $\E$ instead of the entire vector space $V$.
Limiting the search to $\E$ helps us in two different ways. First, it reduces
the complexity of the exhaustive search since the cardinality of $\E$ is smaller
than the cardinality of $V$. Indeed, the sets $\E_j$ are disjoint, so
\begin{align*}
  \Card\E &= \sum_{j=1}^m\Card\E_j\\
  &= \sum_{j=1}^m q^{m-j}\\
  &= \cfrac{q^m-1}{q-1},
\end{align*}
whereas
\[
  \Card V = q^m.
\]
Second, it allows a better understanding of what happens in the algorithm,
because if we have some vector
\[
  a\in\E_j
\]
for a given $1\leq j\leq m$, we know that the associated bilinear form
\[
  (x, y)\mapsto\ps{a}{x}\ps{a}{y}
\]
can only impact the coordinates $i\geq j$. Thus, we further
use the vector space structure of $V$ by searching for solutions
on each coordinate, starting with the first coordinates and vectors in $\E_1$,
then the second coordinate and vectors in $\E_2$, and so on until the last
coordinate. Let us focus on the first coordinate first and give some details.

Recall that the goal is to obtain a trisymmetric decomposition for the
hypersymmetric bilinear map $\Phi:V\times V\to V$, that is written
\[
  \Phi(x, y) = (b_1(x, y), \dots, b_m(x, y)).
\]
in the basis of $V$. We first see how to decompose the bilinear \emph{form}
$b_1$ as a sum of rank $1$ bilinear forms. Let $\B$ be the set of bilinear forms
of $V\times V$, recall that $\B$ is a $\K$-vector space of dimension $m^2$, and
that we identify $b_1$ with the $m\times m$ matrix $B_1\in\K^{m\times m}$ 
such that for all vectors $x, y\in V$, we have
\[
  b_1(x, y) = X B_1 Y^t,
\]
where $X, Y\in\K^{1\times m}$ are the row vectors representing $x$ and $y$ and
where $Y^t$ is the transposed of $y$. Let $r_1$ be the rank of $b_1$, we know
that $b_1$ can be decomposed as a sum of $r_1$ bilinear forms of rank $1$. Let
$f$ be the application mapping an element in $V$ to its associated bilinear
form:
\[
  \begin{array}{cccc}
    f: & V & \to & \B\\
    & a & \mapsto & (x, y)\mapsto\ps{a}{x}\ps{a}{y}.
  \end{array}
\]
In order to find these decompositions, we begin by exhaustively searching through scalars $\lambda_1\in\K$ and vectors
$a_1\in \E_1$ such that
\[
  r_1-1=\rank(b_1-\lambda_1 f(a_1)) < \rank(b_1)=r_1.
\]
Then, for each such couple $(\lambda_1, a_1)$, we exhaustively search through
scalars $\lambda_2\in\K$ and vectors $a_2\in \E_1$ such that
\[
  r_1-2=\rank(b_1-\lambda_1 f(a_1)-\lambda_2f(a_2)) < \rank(b_1-\lambda_1
  f(a_1))=r_1-1.
\]
We continue this process until we have $r_1$ couples $(\lambda_1, a_1), \dots,
(\lambda_{r_1}, a_{r_1})$ such that
\[
  0 = \rank(b_1-\sum_{j=1}^{r_1}\lambda_jf(a_j)) <
  \rank(\sum_{j=1}^{r_1-1}\lambda_jf(a_j)) = 1,
\]
that exactly means that
\[
  b_1 = \sum_{j=1}^{r_1}\lambda_jf(a_j),
\]
and we have found our decomposition.
In fact, we can search in a more clever way. Since the rank of $b_1$ is $r_1$
and we are looking for decompositions as a sum of exactly $r_1$ bilinear forms of
rank $1$, we must choose
couples $(\lambda_j, a_j)$ that decrease the rank of our bilinear form at each
step, otherwise we will need strictly more than $r_1$ bilinear forms of rank
$1$. Therefore, at each step, we can search only through the vectors that can
decrease the rank of the last considered bilinear form. After each choice of
$(\lambda_1, a_1)$, we will thus exhaustively search through scalars
$\lambda_2\in\K$ and vectors $a_2$ in
\[
  \E_1^{\left\{ (\lambda_1, a_1) \right\}} = \left\{
  a\in\E_1\;|\;\exists\lambda\in\K,\,\rank(b_1-\lambda f(a)) <\rank(b_1)
\right\}.
\]
Then, after each choice of $(\lambda_2, a_2)$, we will search through scalars
$\lambda_3\in\K$ and vectors $a_3$ in
\[
  \E_1^{\left\{ (\lambda_1, a_1), (\lambda_2, a_2) \right\}} = \left\{
    a\in\E_1^{ \left\{ (\lambda_1, a_1)
    \right\}}\;|\;\exists\lambda\in\K,\,\rank(b_1-\lambda_1f(a_1)-\lambda f(a))
    <\rank(b_1-\lambda_1f(a_1))
\right\},
\]
and we use the same idea until the end of the process. This strategy, used in
Algorithm~\ref{algo:mindecomp}, permits to
save a lot of time in the search because we look only at the vectors that have
the potential to decrease the rank, instead of all the vectors in $\E_1$.
\begin{algorithm}
  \caption{(Minimal decomposition)}
  \label{algo:mindecomp}
  \begin{algorithmic}[1]
    \Require{$b\in \B$ a bilinear form of rank $r$, $E\subset\E$
    the space of vectors in which to search}
    \Ensure{A list of decompositions of $b$ as a sum of bilinear forms of rank
    $1$, each decomposition represented by a set of $r$ couples $\left\{
      (\lambda_1, a_1), \dots, (\lambda_{r}, a_{r})\right\}$.}

    \Procedure{MinimalDecomposition}{$b, E, R_{\text{glob}},
    R_{\text{loc}}=\emptyset$}
    \If{$\varphi=0$}\Comment{$\rank(\varphi)=0$}
    \State $R_\text{glob}\gets R_\text{glob}\bigcup\left\{R_\text{loc}\right\}$
    \Else
    \State $\mathcal C\gets\emptyset$
    \ForAll{$a\in E$}
    \ForAll{$\lambda\in\K$}
  \If{$\rank(b-\lambda f(a))<\rank(b)$}
    \State $\mathcal C\gets\mathcal C\bigcup\left\{ (\lambda, a) \right\}$
    \State \textbf{break}
    \EndIf
    \EndFor
    \EndFor
    \For{$i=1$ to $\Card\mathcal C$}\Comment{$\mathcal
      C=\left\{ (\lambda_1, a_1), \dots, (\lambda_u, a_u) \right\}$}
      \State $E'\gets\left\{ a_{i+1}, \dots, a_u \right\}$
    \State \Call{FindDecMat}{$b-\lambda_i\f(a_i), E', R_\text{glob},
    R_\text{loc}\cup\left\{ (\lambda_i, a_i) \right\}$}
    \EndFor
    \EndIf
    \EndProcedure

   \State $\mathcal R\gets\emptyset$
    \State \Call{MinimalDecomposition}{$b, E, \mathcal R$}
    \State \Return $\mathcal R$
  \end{algorithmic}
\end{algorithm}

For each decomposition of $b_1$
\[
  b_1(x, y) = \sum_{j=1}^{r_1}\lambda_j\ps{a_j}{x}\ps{a_j}{y}
\]
that we compute this way, we obtain
\[
  \Phi(x, y) - \sum_{j=1}^{r_1}\lambda_j\ps{a_j}{x}\ps{a_j}{y}a_j = (0, b_2'(x,
  y), \dots, b_m'(x, y))
\]
where $b_2', \dots, b_m'\in\B$ are new bilinear forms, depending on the coordinates
of $a_1, \dots, a_{r_1}$, and where the first coordinate is $0$ because the
first coordinate of the vectors $a_1, \dots, a_{r_1}$ is always $1$.

% TODO
% ====
%
% Am example of the coordinate search algorithm that is illuminating: probably
% the example of F_{3^3} because it shows the effect of the margins and the
% number of decompositions is only 4. Also a tree showing that the rank
% decreasing technique allows to eliminate some elements in the search would be
% very nice. Maybe there is someting interesting with F_{5^2}, F_{7^2} also?
% F_{11^2} is already less interesting than F_{3^3} I think.

\subsection{Implementation}
\section{Asymptotic complexities}
