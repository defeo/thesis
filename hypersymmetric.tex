In Chapter~\ref{chap:bilinear}, we have seen the notions of bilinear complexity
and symmetric bilinear complexity. We now investigate even stronger
notions of symmetry, allowing to have very short representations of a bilinear
map.

\minitoc

% TODO
% ====
%
% Find a nice picture to put here to illustrate something in link with the
% chapter.

\clearpage
\section{Symmetric and hypersymmetric fomulas}
% Table of content
% ================
%
% - Recall of the definition of symmetric
% - Existence and lemma for the symmetric case
% - non degenerate bilinear form, link with the trace, but not only
% - Link between symmetric and hypersymmetric in smaller dimension
% - Galois invariance
% - Comment for the case of the particular algebras we study, what is known and
%   what is not
%
% Comment
% =======
%
% Comment about the trisymmetric formulas, it is true that F_4/F_2 can be
% represented by a trisymmetric formula but it is not the case for F_8/F_2. We
% should check the lemma saying something on the existence of the trisymmetric
% decomposition, but it is probably just simpler.

Let $\K$ be a finite field, $V_1$, $V_2$ and $W$ three finite-dimensional $\K$-vector
space and
\[
  \Phi:V_1\times V_2\to W
\]
a bilinear map. Recall Definition~\ref{defi:bilinear-formula}:
\[
  \Phi(x, y) = \sum_{j=1}^t\varphi_j(x)\psi_j(y)w_j,
\]
where for all $1\leq j\leq t$, $\varphi_j\in V_1^\vee$ and $\phi_j\in V_2^\vee$ are linear forms and
$w_j\in W$ is a vector, is called a \emph{bilinear formula} of length $t$. If
the spaces $V_1$ and $V_2$ are equal and if the bilinear map $\Phi$ is
symmetric, \ie if for all $x, y\in V$
\[
  \Phi(x, y) = \Phi(y, x),
\]
we can investigate the existence of formulas satisfying the same condition of
symmetry, \ie formulas where for all $1\leq j\leq t$, $\varphi_j=\psi_j$,
resulting in \emph{symmetric} bilinear form:
\[
  \Phi(x, y) = \sum_{j=1}^t\varphi_j(x)\varphi_j(y)w_j.
\]
In fact, we can define other interesting types of symmetries, but it is useful
to first generalize the notions that we saw in Chapter~\ref{chap:bilinear} to
higher dimensions.
\subsection{Generalization to multilinear maps}
The definitions of bilinear formula and bilinear complexity are not bound to the
bilinear case and can be generalized to arbitrary dimension. These general
definitions will be used in Section~\ref{subsec:trisym} to define
\emph{hypersymmetric} complexity.
\begin{defi}[Multilinear formula]
Let $V_1, V_2, \dots, V_s$ and $W$ be $s+1$ finite-dimensional $\K$-vector
spaces and
\[
  \Phi:V_1\times V_2\times\dots\times V_s\to W
\]
an $s$-linear map. A \emph{multilinear formula}, or \emph{multilinear
decomposition}, or \emph{multilinear algorithm} of length $t$ for $\Phi$ is a
collection of $s\times t$ linear forms $\varphi_1^{(1)}, \varphi_2^{(1)}, \dots,
\varphi_t^{(1)}\in V_1^\vee$ up to $\varphi_1^{(s)}, \varphi_2^{(s)}, \dots,
\varphi_t^{(s)}\in V_s^{\vee}$ and $t$ vectors $w_1, \dots, w_t$, such that for all $x_1\in V_1, \dots, x_s\in
V_s$, we have
\[
  \Phi(x_1, \dots, x_s) =
  \sum_{j=1}^t\varphi_j^{(1)}(x_1)\dots\varphi_j^{(s)}(x_s)w_j.
\]
\end{defi}
\begin{defi}[Multilinear complexity]
Let $V_1, V_2, \dots, V_s$ and $W$ be $s+1$ finite-dimensional $\K$-vector
spaces and
\[
  \Phi:V_1\times V_2\times\dots\times V_s\to W
\]
an $s$-linear map. The \emph{multilinear complexity} $\mu(\Phi)$ of $\Phi$ is the
minimal length $t$ of a multilinear formula for $\Phi$.
\end{defi}
As in the case of bilinear complexity, the multilinear complexity $\mu(\Phi)$ of a
multilinear map $\Phi$ can also be defined as the rank of the tensor in 
\[
  V_1^\vee\otimes\dots\otimes V_s^\vee\otimes W
\]
corresponding to $\Phi$, see Example~\ref{ex:bilinear-complexity} for an
illustration of this correspondence in the bilinear case. In the case where
\[
  V_1 = V_2 = \dots = V_s,
\]
symmetric formulas and
symmetric complexity can also be generalized when $\Phi$ is a \emph{symmetric}
multilinear map, \ie when for all permutation $\sigma\in\mathfrak S_s$ and for
all vectors $x_1, \dots, x_s\in V$, we have
\[
  \Phi(x_1, \dots, x_s) = \Phi(\sigma(x_1), \dots, \sigma(x_s)).
\]
\begin{defi}[Symmetric multilinear formula]
Let $V$ and $W$ be two finite-dimensional $\K$-vector
spaces and
\[
  \Phi:\underset{\textrm{$s$ times}}{\underbrace{V\times\dots\times V}}\to W
\]
a symmetric $s$-linear map. A \emph{symmetric multilinear formula}, or
\emph{symmetric multilinear
decomposition}, or \emph{symmetric multilinear algorithm} of length $t$ for $\Phi$ is a
collection of $t$ linear forms $\varphi_1, \varphi_2, \dots,
\varphi_t\in V^\vee$ and $t$ vectors $w_1, \dots, w_t$, such that for all $x_1, \dots, x_s\in
V$, we have
\[
  \Phi(x_1, \dots, x_s) =
  \sum_{j=1}^t\varphi_j(x_1)\dots\varphi_j(x_s)w_j.
\]
\end{defi}
\begin{defi}[Symmetric multilinear complexity]
Let $V$ and $W$ be two finite-dimensional $\K$-vector
spaces and
\[
  \Phi:\underset{\textrm{$s$ times}}{\underbrace{V\times\dots\times V}}\to W
\]
a symmetric $s$-linear map. The \emph{symmetric multilinear complexity} $\musym(\Phi)$ of $\Phi$ is the
minimal length $t$ of a symmetric multilinear formula for $\Phi$. If no such
formula exists, we set
\[
  \musym(\Phi) = \infty.
\]
\end{defi}
Contrary to the bilinear case, some symmetric multilinear maps do not admit a symmetric
decomposition, but the problem of whether a symmetric multilinear map admits
a symmetric multilinear formula is well understood and follows from
Theorem~\ref{thm:symmetric-formula}.
\begin{thm}[{\cite[Thm.~A.7]{Randriam15}}]\label{th:criterion}
\label{thm:symmetric-formula}
Let $\Phi:V^s\to W$ be a $s$-linear map between finite dimensional vector spaces over $\mathbb{F}_q$.
Then $\Phi$ admits a symmetric decomposition if and only if $\Phi$ is \emph{Frobenius-symmetric},
\ie if and only if it is symmetric and one of the following two conditions holds:
\begin{itemize}
\item $s\leq q$
\item $s\geq q+1$ and for all $u,v,z_1,\dots,z_{s-q-1}$ in $V$,
\[
\Phi(\underset{\textrm{$q$ times}}{\underbrace{u,\dots,u}},v,z_1,\dots,z_{s-q-1})=\Phi(u,\underset{\textrm{$q$ times}}{\underbrace{v,\dots,v}},z_1,\dots,z_{s-q-1}).
\]
\end{itemize}
\end{thm}

\subsection{Trisymmetric and hypersymmetric complexity}
\label{subsec:trisym}

Under even stricter conditions, we can study the existence of even more
symmetric formulas. These formulas allow to describe a multilinear map with
fewer elements, and thus give a compact definition of the map. Since the
symmetry conditions are stronger there are fewer such formulas, and as a
consequence the search space is smaller. Thus, we expect algorithms to be
faster, as was the case when using Barbulescu \etal algorithm
(Algorithm~\ref{algo:BDEZ}) to find symmetric formulas. Let us define those
``stricter conditions''. Let
\[
  \Phi:V^s\to V
\]
be an $s$-linear symmetric map, \ie we additionnaly ask that $W=V$. We also
assume that $V$ has a non-degenerate symmetric bilinear form, that we write as a
scalar product
\[
 \begin{array}{ccc}
 V\times V &\to&\K\\
 (v,w)&\mapsto&\ps{v}{w}.
 \end{array}
\]
In that case, we know that the vector space $V$ is isomorphic to its dual space
$V^\vee$:
\[
  V\cong V^\vee,
\]
\ie for each linear form $\varphi\in V^\vee$, there exist a unique vector $a\in
V$ such that for all $x\in V$, we have
\[
  \varphi(x) = \ps{a}{x}.
\]
Under these conditions, we can now write a symmetric formula for $\Phi$ as
\[
  \Phi(x, y) = \sum_{j=1}^t\ps{a_i}{x}\ps{a_i}{y}w_j
\]
where for all $1\leq j\leq t$, $a_j\in V$ is a vector of $V$. As a consequence,
we can also describe a symmetric formula for $\Phi$ as the data of vectors
$(a_j)_{1\leq j\leq t}$ and $(w_j)_{1\leq j\leq t}$. In order to have an even
more compact description of $\Phi$, one can ask for the vectors $w_j$ to be
proportional to $a_i$, leading to the definition of hypersymmetric
formula.
% Note
% ====
%
% But is that really natural? Wouldn't it be better to say that we would like
% the a_i and the b_i to be equal? 
%
% TODO
% ====
%
% Maybe change this to include the case a_i = b_i and discuss it a little, or
% maybe not, we'll see.

\begin{defi}[Hypersymmetric formula]
Let $V$ a finite-dimensional $\K$-vector
spaces equipped with a saclar product and
\[
  \Phi:\underset{\textrm{$s$ times}}{\underbrace{V\times\dots\times V}}\to W
\]
a symmetric $s$-linear map. A \emph{hypersymmetric formula}, or
\emph{hypersymmetric decomposition}, or \emph{hypersymmetric algorithm} of length $t$ for $\Phi$ is a
collection of $t$ vectors $a_1, \dots, a_t\in V$ and $t$ scalars $\lambda_1,
\dots, \lambda_t\in\K$, such that for all $x_1, \dots, x_s\in
V$, we have
\[
  \Phi(x_1, \dots, x_s) =
  \sum_{j=1}^t\lambda_j\ps{a_j}{x_1}\dots\ps{a_j}{x_s}a_j.
\]
\end{defi}
\begin{defi}[Hypersymmetric complexity]
Let $V$ be a finite-dimensional $\K$-vector spaces equipped with a saclar product
and
\[
  \Phi:\underset{\textrm{$s$ times}}{\underbrace{V\times\dots\times V}}\to W
\]
a symmetric $s$-linear map. The \emph{hypersymmetric complexity} $\muhyp(\Phi)$ of $\Phi$ is the
minimal length $t$ of a hypersymmetric formula for $\Phi$. If no such
formula exists, we set
\[
  \muhyp(\Phi) = \infty.
\]
\end{defi}
\begin{ex}
  We take the same case as in Example~\ref{ex:bilinear-complexity}, but viewed
  a bit differently. Let $\K=\mathbb{F}_2$ and 
  \[
    V=\mathbb{F}_4\cong\mathbb{F}_2[T]/(T^2+T+1)\cong\mathbb{F}_2(\zeta)
  \]
  seen as a $\mathbb{F}_2$-vector space of dimension $2$ using the base $(1,
  \zeta)$. The $\mathbb{F}_2$-bilinear map $\Phi$ that we
  consider is the product in $\mathbb{F}_4$:
  \[
 \begin{array}{cccc}
   \Phi: & \mathbb{F}_4\times \mathbb{F}_4 &\to&\mathbb{F}_4\\
 &(x,y)&\mapsto&xy.
 \end{array}
  \]
  We also consider the non-degenerate symmetric bilinear form
\[
 \begin{array}{ccc}
   \mathbb{F}_4\times \mathbb{F}_4 &\to&\mathbb{F}_2\\
 (v,w)&\mapsto&\tr(vw),
 \end{array}
\]
where $\tr$ is the trace of the field extension $\mathbb{F}_4/\mathbb{F}_2$,
and we write
\[
  \tr(xy) = \ps{x}{y}.
\]
If $x = x_0 + x_1\zeta\in\mathbb{F}_4$ is an element in the extension field, we have $\tr(x) = x_1$, and if $y = y_0
+ y_1\zeta\in\mathbb{F}_4$ is another element, then their product is
\[
  xy = x_0y_0 + x_1y_1 + (x_0y_1 + x_1y_0 + x_1y_1)\zeta.
\]
We also see that
\[
\left\{ 
  \begin{array}{lll}
    \ps{1}{x}\ps{1}{y} &=& x_1y_1 \\
    \ps{1+\zeta}{x}\ps{1+\zeta}{y} &=& x_0y_0 \\
    \ps{\zeta}{x}\ps{\zeta}{y} &=& (x_0+x_1)(y_0+y_1)
  \end{array}
\right.
\]
and thus we have
\[
  xy =
  \ps{1}{x}\ps{1}{y}\cdot1+\ps{1+\zeta}{x}\ps{1+\zeta}{y}\cdot(1+\zeta)+\ps{\zeta}{x}\ps{\zeta}{y}\cdot\zeta.
\]
This is an hypersymmetric formula of length $3$, and we can prove that there are
no formulas of length $2$, so we have
\[
  \muhyp(\Phi) = 3.
\]
This is in fact the very same formula as in
Example~\ref{ex:bilinear-complexity}.
\end{ex}
In order to investigate the existence of hypersymmetric decompositions, we
remark that there is a natural link between hypersymmetric decompositions of 
the $s$-linear map
\[
  \Phi:V^s\to V
\]
and symmetric decompositions of the $(s+1)$-linear form $\widetilde\Phi$ defined by
\[
  \begin{array}{llll}
    \widetilde\Phi:&V^{s+1}&\to&\K\\
    &(x_1, \dots, x_{s+1})&\mapsto&\ps{\Phi(x_1, \dots, x_s)}{x_{s+1}}
  \end{array}
\]
in the sense of~Lemma~\ref{lm:link-hyp-sym}. Moreover, we say that the
$s$-linear map $\Phi$ is \emph{hypersymmetric} if the associated $(s+1)$-linear
form $\widetilde\Phi$ is symmetric.
\begin{lm}
  \label{lm:link-hyp-sym}
  Let $V$ a $\K$-vector space and 
  \[
    \Phi:V^s\to V
  \]
  a symmetric $s$-linear map. 
Elements $(a_j)_{1\leq j\leq t}$ in $V$ and scalars $(\lambda_j)_{1\leq j\leq t}$ in $\K$ define a hypersymmetric formula for the $s$-linear map $\Phi$,
\[
\Phi(x_1,\dots,x_s)=\sum_{j=1}^{t}\lambda_j\ps{a_j}{x_1}\cdots\ps{a_j}{x_t}a_j,
\]
if and only if they define a symmetric formula for the $(s+1)$-linear form $\widetilde{\Phi}$,
\[
\widetilde{\Phi}(x_1,\dots,x_s,x_{s+1})=\sum_{j=1}^{t}\lambda_i\ps{a_j}{x_1}\cdots\ps{a_j}{x_t}\ps{a_j}{x_{s+1}}.
\]

Thus, $\Phi$ admits a hypersymmetric formula if and only if $\widetilde{\Phi}$ is Frobenius-symmetric (in the sense of Theorem~\ref{thm:symmetric-formula}),
and we have
\[
\muhyp(\Phi)=\musym\left(\widetilde{\Phi}\right).
\]

In particular, if $q\geq s+1$, then any hypersymmetric $s$-linear map over $\mathbb{F}_q$ admits a hypersymmetric formula.
\end{lm}
\begin{proof}
  Assume that $\Phi$ admits a hypersymmetric decomposition, such that for all
  $x_1, \dots, x_s\in V$, we have
  \[
    \Phi(x_1,\dots,x_s)=\sum_{j=1}^{t}\lambda_i\ps{a_j}{x_1}\cdots\ps{a_j}{x_t}a_j,
  \]
  then, by taking the scalar product with any $x_{s+1}$, we obtain
\[
  \ps{\Phi(x_1, \dots,
  x_s)}{x_{s+1}}=\widetilde{\Phi}(x_1,\dots,x_s,x_{s+1})=\sum_{j=1}^{t}\lambda_i\ps{a_j}{x_1}\cdots\ps{a_j}{x_t}\ps{a_j}{x_{s+1}},
\]
which defines a symmetric decomposition for $\widetilde\Phi$. In the other
direction, assume that $\widetilde\Phi$ admits a symmetric decomposition, such
that for all $x_1, \dots, x_{s+1}\in V$, we have
\[
\widetilde{\Phi}(x_1,\dots,x_s,x_{s+1})=\sum_{j=1}^{t}\lambda_i\ps{a_j}{x_1}\cdots\ps{a_j}{x_t}\ps{a_j}{x_{s+1}}.
\]
It can also be written as
\[
  \ps{\Phi(x_1, \dots,
  x_s)}{x_{s+1}}=\ps{\sum_{j=1}^t\lambda_j\ps{a_j}{x_1}\cdots\ps{a_j}{x_s}a_j}{x_{s+1}},
\]
so that we have
\[
  \ps{\Phi(x_1, \dots,
  x_s)-\sum_{j=1}^t\lambda_j\ps{a_j}{x_1}\cdots\ps{a_j}{x_s}}{x_{s+1}}=0.
\]
Since the scalar product $\ps{\cdot}{\cdot}$ is non-degenerate, it means that
  \[
    \Phi(x_1,\dots,x_s)=\sum_{j=1}^{t}\lambda_i\ps{a_j}{x_1}\cdots\ps{a_j}{x_t}a_j.
  \]
  Hence $\Phi$ admits a hypersymmetric decomposition. The other assertions
  follow.
\end{proof}
The most important case is arguably the bilinear case, where $s=2$, because it
was thoroughly studied. For that reason, we sometimes replace the word
hypersymmetric by \emph{trisymmetric} in that particular case, because of the
form of the formulas
\[
  \Phi(x, y) = \sum_{j=1}^t\lambda_j\ps{a_j}{x}\ps{a_j}{y}a_j
\]
that includes the same element $a_j$ three times, and we write $\mutri(\Phi)$
instead of $\muhyp(\Phi)$. Lemma~\ref{lm:link-hyp-sym} states that, if $q\geq3$, a
trisymmetric map $\Phi$ always admits a trisymmetric decomposition.

\subsection{Galois invariance}

An other type of interesting decompositions is Galois invariant decompositions.
It is motivated by the study of group actions on the set of decompositions, that
can sometimes be used to cut branches in the search tree of the algorithms.
% TODO
% ====
%
% Link that with BDEZ stab, the version using automorphisms, and Covanov.
Let 
\[
 \begin{array}{cccc}
   \sigma: & V\times V &\to&V\\
 &x&\mapsto&x^\sigma
 \end{array}
\]
be a $\K$-automorphism of $V$ that respects the scalar product, \ie for all $x,
y\in V$, we have
\[
  \ps{x^\sigma}{y^\sigma} = \ps{x}{y}.
\]
Then, if $\sigma$ is also compatible with some studied multilinear map $\Phi$, it induces
an action on the set of decomposition, as explained in
Lemma~\ref{lm:action-sym}.
\begin{lm}
  \label{lm:action-sym}
  Let $V$ a finite-dimensional $\K$-vector space and
  \[
    \Phi:V^s\to V
  \]
  a symmetric $s$-linear map that is compatible with $\Phi$, \ie for all $x_1,
  \dots, x_s$ in $V$, we have
  \[
    \Phi(x_1^\sigma, \dots, x_s^\sigma) = \Phi(x_1, \dots, x_s)^\sigma.
  \]
  If $(a_j)_{1\leq j \leq t}$ and $(b_j)_{1\leq j \leq t}$ define a symmetric
  formula for $\Phi$
  \[
    \Phi(x_1, \dots, x_s) = \sum_{j=1}^t\ps{a_j}{x_1}\dots\ps{a_j}{x_s}b_j,
  \]
  then $(a_j^\sigma)_{1\leq j\leq t}$ and $(b_{j}^\sigma)_{1\leq j\leq t})$ also
  define a symmetric formula for $\Phi$
  \[
    \Phi(x_1, \dots, x_s) =
    \sum_{j=1}^t\ps{a_j^\sigma}{x_1}\dots\ps{a_j^\sigma}{x_s}b_j^\sigma.
  \]
\end{lm}
\begin{proof}
 Assume that we have a symmetric decomposition for $\Phi$, with the same
 notations as in the Lemma. First, notice that for every $x,y\in V$, we have
 \[
   \ps{x^\sigma}{y} = \langle{x},{y^{\sigma^{-1}}}\rangle.
 \]
 Then, it follows that
 \begin{align*}
   \Phi(x_1, \dots, x_s) &= \Phi(x_1^{\sigma^{-1}}, \dots,
   x_{s}^{\sigma^{-1}})^\sigma\\
   &=
   (\sum_{j=1}^t\langle{a_j},{x_1^{\sigma^{-1}}}\rangle\dots\langle{a_j},{x_s^{\sigma^{-1}}}\rangle
   b_j)^\sigma\\
   &= \sum_{j=1}^t\ps{a_j^{\sigma}}{x_1}\dots\ps{a_j^\sigma}{x_s}b_j^\sigma.
 \end{align*}
 Thus we have a new symmetric formula for $\Phi$.
\end{proof}

\section{Algorithmic seach in small dimension}
\section{Asymptotic complexities}
