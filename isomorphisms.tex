We studied in Part~\ref{part:single} the arithmetic of a single finite field
extension. We now study a set of several extensions. The very first step will be
to understand how to compute an isomorphism (or an embedding) between two finite
fields: that is the material of this chapter.
\minitoc

% TODO: figure

\clearpage

Our reference for this chapter is~\cite{BDDFS17}: we cover a subpart of the
paper because we are interested in the naive isomorphism algorithm (used in
Chapter~\ref{chap:lattice}) and Allombert's algorithm (used in
Chapter~\ref{chap:standard}). We thus do not cover all the isomorphism
algorithms, the reader interested in Rains' algorithm and its elliptic variant
can take a look at the paper cited above.

\section{Generalities and naive algorithm}

Even if our real goal is to compute \emph{embeddings} of finite fields, \ie
ring homomorphisms
\[
  \phi:K\to L
\]
with $K$ and $L$ finite fields, we often refer to the algorithms as
\emph{isomorphism} algorithms. Indeed, computing the embedding $\phi$ is
the same as computing an isomorphism
\[
  \phi':K\to K'
\]
where $K\cong K'$ is isomorphic to a subfield $K'\subset L$ of $L$. The
isomorphism $\phi'$ is just the embedding $\phi$ with its codomain being
restricted to $K'$.

\subsection{Description of the problem}

We let $q$ be a prime power, $\K = \mathbb{F}_q$ be the field with $q$ elements,
and $f, g\in\K[X]$ two irreducible polynomials with
\[
  m=\deg f\,|\,\deg g=n.
\]
Let
\[
  K=\K[X]/(f(X))\cong\mathbb{F}_{q^m}
\]
and
\[
  L = \K[Y]/(g(Y))\cong\mathbb{F}_{q^n}
\]
two extensions of $\K$. We know there is an embedding
\[
  \phi:K\to L,
\]
unique up to $\K$-automorphism of $K$, \ie there are
\[
  \Card\Gal(K/\K)=m
\]
different embeddings from $K$ to $L$, than can be described as
\[
  \phi\circ\sigma
\]
for $\sigma\in\Gal(K/\K)$. Equivalently, they can also be described as
\[
  \sigma'\circ\phi
\]
with $\sigma'\in\Gal(\phi(K)/\K)$. The \emph{embedding problem} is then to
efficiently find, represent and evaluate one such embedding $\phi$. The problem
is split in two sub-parts.
\begin{description}
  \item[Embedding description problem.] Compute elements $\alpha\in K$ and
    $\beta\in L$ such that 
    \[
      K=\K(\alpha)
    \]
    and such that there exists an embedding $\phi$ mapping $\alpha$ to $\beta$.
  \item[Embedding evaluation problem.] Given elements $\alpha$ and $\beta$
    defined above, and elements $\gamma\in K$, $\delta\in L$, solve the
    following problems:
    \begin{itemize}
      \item compute $\phi(\gamma)\in L$;
      \item test if $\delta\in\phi(K)$;
      \item if $\delta\in\phi(K)$, then compute $\phi^{-1}(\delta)\in K$.
    \end{itemize}
\end{description}
As the name suggests, the \emph{embedding description problem} focuses on
finding a pair of elements that are sufficient to describe an embedding. Indeed,
if 
\[
  K=\K(\alpha)
\]
we know that every element $x\in K$ can be uniquely written as 
\[
  x = \sum_{j=0}^{m-1}a_j\alpha^j
\]
with $a_j\in\K$ for al $0\leq j\leq m-1$, and the embedding $\phi$ is then
defined by
\[
  \phi(x) = \sum_{j=0}^{m-1}a_j\beta^j.
\]
\begin{prop}
  \label{prop:description}
 The elements $\alpha$ and $\beta$
 describe an embedding if and only if they have the same minimal polynomial. 
\end{prop}
\begin{proof}
  Let $\phi:K\to L$ be an embedding mapping $\alpha$ to $\beta$ and let 
  \[
    P = \Minpoly_\K(\alpha)
  \]
  be the the minimal polynomial of $\alpha$. Then 
  \begin{align*}
    P(\beta) &= P(\phi(\alpha)) \\
    &= \phi(P(\alpha))\\
    &= \phi(0) \\
    &= 0
  \end{align*}
  thus $\Minpoly_\K(\beta)\neq 1$ divides $P$ which is irreducible so 
  \[
\Minpoly_\K(\beta) = P.
  \]
 Conversely, if $\alpha$ and $\beta$ have the same minimal polynomial $P$, then the
 map $\phi$ is well-defined and defines an isomorphism between the fields $\K(\alpha)$ and
 $\K(\beta)$, that are both isomorphic to the field
 \[
   \K[X]/(P(X)).
 \]
\end{proof}
While the first problem focuses on finding a description of $\phi$, the
\emph{embedding evaluation problem} independently asks how to efficiently use
the description to compute the actual embedding. We target this question in
Section~\ref{sec:evaluation}.

\subsection{Embedding description problem and naive algorithm}

Until the end of this section and in Section~\ref{sec:allombert}, we deal with
the \emph{embedding description problem}, although we only review a subpart of
the existing algorithms (see~\cite{BDDFS17} for other algorithms). As above, let
$f$ and $g$ two irreducible polynomials with coefficients in $\K$ such that
\[
  m=\deg f\,|\,\deg g=n.
\]
and let
\[
  K=\K[X]/(f(X))\cong\mathbb{F}_{q^m}
\]
and
\[
  L = \K[Y]/(g(Y))\cong\mathbb{F}_{q^n}
\]
two finite fields. Then one can simply take $\alpha$ to be the class of $X$ in
$K$ and choose $\beta$ to be any root of $f$ in $L$. Indeed, we know that there
is an isomorphic copy of $K$ in $L$ and thus that $f$ splits over $L$.
Furthermore, any root of $f$ will have $f$ as its minimal polynomial, which is
also the minimal polynomial of $\alpha$ by construction. By
Proposition~\ref{prop:description}, the map 
\[
  \phi:K\to L
\]
sending $\alpha$ to $\beta$ is an embedding. The critical routine in that
algorithm is to find a root of $f$ in $L$, that can be done using Shoup-Kaltofen
\emph{equal degree factorization} algorithm~\cite{KS97}. The complexity analysis
of~\cite{BDDFS17} indicates that the cost is stricly larger than quasi-quadratic
complexity $\tilde O(m^2)$. A more efficient algorithm, due to Lenstra and
Allombert, is discussed in Section~\ref{sec:allombert}.

\section{Lenstra-Allombert algorithm}
\label{sec:allombert}

Both Lenstra~\cite{Lenstra91} and Allombert used
Kummer theory, the study of certain field extensions, to compute isomorphisms
between finite fields. But while Lenstra's focus
was on proving the existence of a deterministic isomorphism algorithm, Allombert 
wanted to provide a practical algorithm. This led to the invention of the
Lenstra-Allombert algorithm~\cite{Allombert02} in 2002, for which we give a
description in this section. The ideas of Allombert play an important part in
Chapter~\ref{chap:standard} too.

\section{The embedding evaluation problem}
\label{sec:evaluation}

