We studied in Part~\ref{part:single} the arithmetic of a single finite field
extension. We now study a set of several extensions. The very first step will be
to understand how to compute an isomorphism (or an embedding) between two finite
fields: that is the material of this chapter.
\minitoc

% TODO: figure

\clearpage

Our reference for this chapter is~\cite{BDDFS17}: we cover a subpart of the
paper because we are interested in the naive isomorphism algorithm (used in
Chapter~\ref{chap:lattice}) and Allombert's algorithm (used in
Chapter~\ref{chap:standard}). We thus do not cover all the isomorphism
algorithms, the reader interested in Rains' algorithm and its elliptic variant
can take a look at the paper cited above.

\section{Generalities and naive algorithm}

Even if our real goal is to compute \emph{embeddings} of finite fields, \ie
ring homomorphisms
\[
  \phi:K\to L
\]
with $K$ and $L$ finite fields, we often refer to the algorithms as
\emph{isomorphism} algorithms. Indeed, computing the embedding $\phi$ is
the same as computing an isomorphism
\[
  \phi':K\to K'
\]
where $K\cong K'$ is isomorphic to a subfield $K'\subset L$ of $L$. The
isomorphism $\phi'$ is just the embedding $\phi$ with its codomain being
restricted to $K'$.

\subsection{Description of the problem}

We let $p$ be a prime number, $\K = \mathbb{F}_p$ be the field with $p$ elements,
and $f, g\in\K[X]$ two irreducible polynomials with
\[
  m=\deg f\mid\deg g=n.
\]
Let
\[
  K=\K[X]/(f(X))\cong\mathbb{F}_{p^m}
\]
and
\[
  L = \K[Y]/(g(Y))\cong\mathbb{F}_{p^n}
\]
two extensions of $\K$. We know there is an embedding
\[
  \phi:K\to L,
\]
unique up to $\K$-automorphism of $K$, \ie there are
\[
  \Card\Gal(K/\K)=m
\]
different embeddings from $K$ to $L$, than can be described as
\[
  \phi\circ\sigma
\]
for $\sigma\in\Gal(K/\K)$. Equivalently, they can also be described as
\[
  \sigma'\circ\phi
\]
with $\sigma'\in\Gal(\phi(K)/\K)$. The \emph{embedding problem} is then to
efficiently find, represent and evaluate one such embedding $\phi$. The problem
is split in two sub-parts.
\begin{description}
  \item[Embedding description problem.] Compute elements $\alpha\in K$ and
    $\beta\in L$ such that 
    \[
      K=\K(\alpha)
    \]
    and such that there exists an embedding $\phi$ mapping $\alpha$ to $\beta$.
  \item[Embedding evaluation problem.] Given elements $\alpha$ and $\beta$
    defined above, and elements $\gamma\in K$, $\delta\in L$, solve the
    following problems:
    \begin{itemize}
      \item compute $\phi(\gamma)\in L$;
      \item test if $\delta\in\phi(K)$;
      \item if $\delta\in\phi(K)$, then compute $\phi^{-1}(\delta)\in K$.
    \end{itemize}
\end{description}
As the name suggests, the \emph{embedding description problem} focuses on
finding a pair of elements that are sufficient to describe an embedding. Indeed,
if 
\[
  K=\K(\alpha)
\]
we know that every element $x\in K$ can be uniquely written as 
\[
  x = \sum_{j=0}^{m-1}a_j\alpha^j
\]
with $a_j\in\K$ for al $0\leq j\leq m-1$, and the embedding $\phi$ is then
defined by
\[
  \phi(x) = \sum_{j=0}^{m-1}a_j\beta^j.
\]
\begin{prop}
  \label{prop:description}
 The elements $\alpha$ and $\beta$
 describe an embedding if and only if they have the same minimal polynomial. 
\end{prop}
\begin{proof}
  Let $\phi:K\to L$ be an embedding mapping $\alpha$ to $\beta$ and let 
  \[
    P = \Minpoly_\K(\alpha)
  \]
  be the the minimal polynomial of $\alpha$. Then 
  \begin{align*}
    P(\beta) &= P(\phi(\alpha)) \\
    &= \phi(P(\alpha))\\
    &= \phi(0) \\
    &= 0
  \end{align*}
  thus $\Minpoly_\K(\beta)\neq 1$ divides $P$ which is irreducible so 
  \[
\Minpoly_\K(\beta) = P.
  \]
 Conversely, if $\alpha$ and $\beta$ have the same minimal polynomial $P$, then the
 map $\phi$ is well-defined and defines an isomorphism between the fields $\K(\alpha)$ and
 $\K(\beta)$, that are both isomorphic to the field
 \[
   \K[X]/(P(X)).
 \]
\end{proof}
While the first problem focuses on finding a description of $\phi$, the
\emph{embedding evaluation problem} independently asks how to efficiently use
the description to compute the actual embedding. We target this question in
Section~\ref{sec:evaluation}.

\subsection{Embedding description problem and naive algorithm}

Until the end of this section and in Section~\ref{sec:allombert}, we deal with
the \emph{embedding description problem}, although we only review a subpart of
the existing algorithms (see~\cite{BDDFS17} for other algorithms). As above, let
$f$ and $g$ two irreducible polynomials with coefficients in $\K$ such that
\[
  m=\deg f\mid \deg g=n.
\]
and let
\[
  K=\K[X]/(f(X))\cong\mathbb{F}_{p^m}
\]
and
\[
  L = \K[Y]/(g(Y))\cong\mathbb{F}_{p^n}
\]
two finite fields. Then one can simply take $\alpha$ to be the class of $X$ in
$K$ and choose $\beta$ to be any root of $f$ in $L$. Indeed, we know that there
is an isomorphic copy of $K$ in $L$ and thus that $f$ splits over $L$.
Furthermore, any root of $f$ will have $f$ as its minimal polynomial, which is
also the minimal polynomial of $\alpha$ by construction. By
Proposition~\ref{prop:description}, the map 
\[
  \phi:K\to L
\]
sending $\alpha$ to $\beta$ is an embedding. The critical routine in that
algorithm is to find a root of $f$ in $L$, that can be done using Shoup-Kaltofen
\emph{equal degree factorization} algorithm~\cite{KS97}. The complexity analysis
of~\cite{BDDFS17} indicates that the cost is strictly larger than quasi-quadratic
complexity $\tilde O(m^2)$. A more efficient algorithm, due to Lenstra and
Allombert, is discussed in Section~\ref{sec:allombert}.

\section{Lenstra-Allombert algorithm}
\label{sec:allombert}

Both Lenstra~\cite{Lenstra91} and Allombert used
Kummer theory, the study of certain field extensions, to compute isomorphisms
between finite fields. But while Lenstra's focus
was on proving the existence of a deterministic isomorphism algorithm, Allombert 
wanted to provide a practical algorithm. This led to the invention of the
Lenstra-Allombert algorithm~\cite{Allombert02} in 2002, for which we give a
description in this section. The ideas of Allombert play an important part in
Chapter~\ref{chap:standard} too. The techniques based on Kummer theories work
for extensions of degree $n$ coprime to the characteristic $p$. In order to have
an algorithm working for any type of extension, the solution is to deal with the
part of the extension which degree is divisible by $p$ separately using
Artin-Shreier theory and to glue the results together in the end.

% TODO
% ====
%
% Add a part on Artin-Shreier?


\subsection{Preliminaries}
\label{sec:preliminaries}

Let us first discuss a simpler case than the general one, that will highlight
the method behind Lenstra-Allombert isomorphism algorithm. Let $K$ and $L$ be
two finite fields of cardinality $p^n$, such that
\[
  K\cong L\cong \mathbb{F}_{p^n}.
\]
Assume that $\gcd(p, n)=1$ and that
\[
  n\mid p-1,
\]
or equivalently that there is a primitive $n$-th root of unity in
$\K=\mathbb{F}_p$, that we denote by $\zeta$. The algorithm is based on
Proposition~\ref{prop:h90}.

\begin{prop}[Hilbert $90$ theorem]
  \label{prop:h90}
  Let $K$ be a finite extension of $\K=\mathbb{F}_p$ of degree $n$ such that there exists a
primitive $n$-th root of unity $\zeta\in\K$ in the base field $\K$, \ie such
that $n$ divides $p-1$. 
 Let $\sigma$ be the generator of the Galois group of the extension
 \[
   K/\K
 \]
 and consider the following equation in $K$.
 \begin{equation}
   \tag{H90}
   \sigma(x) = \zeta x
   \label{eq:h90}
 \end{equation}
The solutions of~\eqref{eq:h90} form a one dimensional $\K$-vector space and if
$\alpha\in K$ is such a solution, we have
\[
  \alpha^n\in\K.
\]
If $\alpha$ is also nonzero, then it is a generator of $K$ over $\K$.
\end{prop}
\begin{proof}
  Let us first construct a nonzero solution of~\eqref{eq:h90}. Consider the polynomial
  \[
    P = \sum_{j=0}^{n-1}\zeta^{-j} X^{p^j}
  \]
  of degree $p^{n-1}$. The polynomial $P$ has at most $p^{n-1}$ roots in $K$,
  which has cardinality $p^n$, so there exists some element $x\in K$ such that
  \[
    y = P(x)\neq0.
  \]
  Now, by construction, we have
  \begin{align*}
    \sigma(y) &= \sigma(\sum_{j=0}^{n-1}\zeta^{-j}x^{\sigma^{j}})\\
    &= \sum_{j=0}^{n-1}\zeta^{-j}x^{\sigma^{j+1}}\\
    &= \zeta \times \sum_{j=0}^{n-1}\zeta^{-(j+1)}x^{\sigma^{j+1}}\\
    &= \zeta \times \sum_{j=1}^{n}\zeta^{-j}x^{\sigma^{j}}\\
    &= \zeta y
  \end{align*}
  and thus $y$ is a nonzero solution of~\eqref{eq:h90}. All the elements
  \[
    \lambda y
  \]
  with $\lambda\in\K$ are also solution of~\eqref{eq:h90} since
  \[
    \sigma(\lambda y) = \lambda\sigma(y) = \zeta\lambda y,
  \]
  and the equation has at most $p$ solutions because the polynomial
  \[
    X^p - \zeta X
  \]
  has at most $p$ roots in $K$. Thus there are exactly $p$ different solutions,
  that are the elements of $\Vect(y)$. Let $z$ be a solution of~\eqref{eq:h90},
  then we have
  \begin{align*}
   \sigma(z^n) &= \sigma(z)^n\\
   &= (\zeta z)^n\\
   &= z^n,
  \end{align*}
  therefore $z^n$ is fixed by $\sigma$, which means that
  \[
    z^n\in\K.
  \]
  If $z$ is also nonzero, then for all $0\leq j<n$, we have
  \begin{align*}
    \sigma^j(z) &= \underbrace{(\sigma\circ\dots\circ\sigma)}_{j\text{ times}}(z)\\
    &= \underbrace{(\sigma\circ\dots\circ\sigma)}_{j-1\text{ times}}(\zeta z)\\
    &= \zeta\underbrace{(\sigma\circ\dots\circ\sigma)}_{j-1\text{ times}}(z)\\
    &= \zeta^j z\\
    &\neq z.
  \end{align*}
  Consequently, $z$ is not in any subfield of $K$ and is thus a generator of $K$
  over $\K$.
\end{proof}
Note that Proposition~\ref{prop:h90} applies both to the fields $K$ and $L$,
therefore we can solve Equation~\eqref{eq:h90} in both fields. Let $\alpha_K$ be a
solution of Equation~\eqref{eq:h90} for the root $\zeta$ in $K$, and $\alpha_L$
a solution in $L$. Since we want the primitive $n$-th root of unity $\zeta$ to
be the same in $K$ and $L$, we assume that we already have an embedding from
$\K$ in both these fields. In practice, since $\K$ is a
prime field and the fields $K$ and $L$ are represented by polynomials over
$\K = \mathbb{F}_p=\mathbb{Z}/p\mathbb{Z}$, the assumption is not really
hard to meet. Let
\[
  a_K = \alpha_K^n
\]
and
\[
  a_L = \alpha_L^n.
\]
By Proposition~\ref{prop:h90}, we know that $a_K$ and $a_L$ are both in $\K$ and
this can be used to compute an isomorphism between $K$ and $L$.
\begin{prop}[Allombert~{\cite{Allombert02}}]
 The quotient
 \[
   a_K/a_L
 \]
 is an $n$-th power in $\K$, and if
 \[
   c^n = a_K/a_L
 \]
 then the map sending $\alpha_K$ to $c\alpha_L$ is an isomorphism from $K$ to $L$.
\end{prop}
\begin{proof}
  Let $\phi:K\to L$ be a $\K$-isomorphism between $K$ and $L$. We have
  \begin{align*}
    \sigma(\phi(\alpha_K)) &= \phi(\alpha_K)^p\\
    &= \phi(\alpha_K^p)\\
    &= \phi(\sigma(\alpha_K))\\
    &= \phi(\zeta\alpha_K)\\
    &= \zeta\phi(\alpha_K)
  \end{align*}
  thus $\phi(\alpha_K)$ is a solution of~\eqref{eq:h90} and by
  Proposition~\ref{prop:h90} there exists $\lambda\in\K$ such that
  \[
    \phi(\alpha_K) = \lambda\alpha_L.
  \]
  We also have
  \[
    \phi(\alpha_K)^n = \phi(\alpha_K^n) = \phi(a_K) = a_K,
  \]
  therefore the quotient
  \begin{align*}
    a_K/a_L &= \phi(\alpha_K)^n/\alpha_L^n\\
    &= \lambda^{n}
  \end{align*}
  is an $n$-th power in $\K$. Now let $c\in\K$ be any $n$-th root of $a_K/a_L$,
  then
  \[
    c = \zeta^j\lambda
  \]
  for some $0\leq j\leq n-1$, that is $c$ and $\lambda$ differ by a $n$-th root
  of unity, and so do $\phi(\alpha_K)$ and $c\alpha_L$:
  \[
    c\alpha_L = \zeta^j\phi(\alpha_K) = \sigma^j(\phi(\alpha_K)).
  \]
  Finally, the elements $c\alpha_L$ and $\phi(\alpha_K)$ have the same minimal
  polynomial, because they are conjugates, and $\phi(\alpha_K)$ has the same
  minimal polynomial as $\alpha_K$ because $\phi$ is an isomorphism, so by
  Proposition~\ref{prop:description}, the map sending $\alpha_K$ to $c\alpha_L$ is an
  isomorphism from $K$ to $L$.
\end{proof}
In this simpler case ($n$ divides $p-1$), Lenstra-Allombert algorithm consists
in
\begin{enumerate}
  \item finding $\alpha_K\in K$ and $\alpha_L\in L$ with
    $\sigma(\alpha_K)=\zeta\alpha_K$ and $\sigma(\alpha_L)=\zeta\alpha_L$;
  \item computing a $n$-th $c$ root of $\alpha_K^n/\alpha_L^n$;
  \item returning the embedding described by $\alpha_K\mapsto c\alpha_L$.
\end{enumerate}

\paragraph{General case.} When $n\nmid p-1$, which is always
the case asymptotically since $n>p-1$ at some point, there are no $n$-th roots
of unity in $\K$, and the strategy of Section~\ref{sec:preliminaries} cannot be
applied as if. Nevertheless, it is still possible to apply a similar idea by
extending the space so that it contains roots of unity.

\subsection{Kummer algebras}
\label{sec:kummer-algebras}

% TODO
% ====
%
% Fix the definitions by saying something about where do we pick the roots of
% unity from. Algebraic closure ? Doesn't that make some of the results trivial or
% something? 

Instead of ``just'' working in $\mathbb{F}_{p^n}$, we work in
\[
  A_n = \mathbb{F}_{p^n}\otimes \mathbb{F}_p(\zeta),
\]
where $\zeta$ is a primitive $n$-th root of unity, and where $\otimes$ is the tensor
product over $\K=\mathbb{F}_p$. We thus extend the scalars and force the existence of
suitable roots of unity. The $\K$-algebra $A_n$, that we call \emph{Kummer
algebra}, can now be used instead of
$\mathbb{F}_{p^n}$ in Lenstra-Allombert algorithm.

\begin{defi}[Kummer algebra]
 We call the $\K$-algebra
 \[
   A_n = \mathbb{F}_{p^n}\otimes\mathbb{F}_{p}(\zeta)
 \]
 a \emph{Kummer algebra of degree $n$}.
\end{defi}
\begin{defi}[Field of scalars]
  Let $A_n$ be a Kummer algebra of degree $n$. Then we define
  $\mathbb{F}_{p}(\zeta)$ as the \emph{field of scalars} of $A_n$, and we
  define the \emph{level} $\nu(n)$ of $A_n$ as
  \[
    \nu(n) = \mathrm{ord}_{(\mathbb{Z}/n\mathbb{Z})^\times}(p) = \left[
      \mathbb{F}_{p}(\zeta):\K \right],
  \]
  that is the degree of its field of scalars.
\end{defi}
Let $\sigma:x\mapsto x^p$ be the Frobenius automorphism of the extension
\[
  \mathbb{F}_{p^n}/\K,
\]
we extend it to $A_n$ by defining the linear map
\[
  \begin{array}{cccc}
    \sigma\otimes 1: & A_n & \to & A_n\\
    & \sum_j x_j\otimes y_j & \mapsto & \sum_j \sigma(x_j) \otimes y_j.
  \end{array}
\]
The map $\sigma\otimes 1$ will play the role of $\sigma$ in the simpler case.
\begin{lm}
  The map $\sigma\otimes1$ is a $1\otimes\mathbb{F}_{p}(\zeta)$-linear
  endomorphism with $n$ distinct eigenvalues, that are the powers of
  $1\otimes\zeta$.
\end{lm}
\begin{proof}
  Because of the linear independance of characters~\cite[Chapter VI,
  §4]{Lang04}, we know that the automorphisms
  \[
    \Id, \sigma, \sigma^2, \dots, \sigma^{n-1}
  \]
  are independant. We also know that 
  \[
    \sigma^n = \Id,
  \]
  therefore the minimal polynomial of the $\K$-linear endomorphism $\sigma$ is
  \[
    X^n-1.
  \]
  By the Cayley-Hamilton theorem, we deduce that the characteristic
  polynomial of $\sigma$ is also $X^n-1$. Now let $\B=\left\{ b_1, \dots, b_n \right\}$ be a basis of
  $\mathbb{F}_{p^n}/\K$ and let $M$ be the matrix of $\sigma$ in this basis.
  Then the matrix of the $1\otimes\mathbb{F}(\zeta)$-linear endomorphism
  $\sigma\otimes1$ in the basis
  \[
    \B\otimes 1 =\left\{ b_1\otimes1, \dots, b_n\otimes1 \right\}
  \]
  is also $M$. Thus, the characteristic polynomial of $\sigma\otimes1$ is again
  $X^n-1$, that splits completely in 
  \[
    1\otimes\mathbb{F}_{p}(\zeta)\cong \mathbb{F}_{p}(\zeta),
  \]
  the roots being the elements
  \[
    1\otimes\zeta^j
  \]
  for $0\leq j\leq n-1$. Finally, the eigenvalues are the roots of the
  characteristic polynomial so this concludes the proof.
\end{proof}
Since there are exactly $n$ distinct eigenvalues, we know that the
corresponding eigenspaces are all one-dimensional
$1\otimes\mathbb{F}_{p}(\zeta)$-vector spaces, and the eigenspace corresponding
to the eigenvalue $\zeta$ is described by the equation
 \begin{equation}
   \tag{H90}
   (\sigma\otimes1)(x) = (1\otimes\zeta) x
   \label{eq:h90-kummer}
 \end{equation}
 that we again denote by~\eqref{eq:h90-kummer}.
 \begin{lm}
   Let $\alpha$ be a nonzero solution of~\eqref{eq:h90-kummer} for the root
   $\zeta$. Then 
   \[
     \alpha^n\in 1\otimes\mathbb{F}_{p}(\zeta)
   \]
   and $\alpha$ is a generating element for $A_n$ as an algebra over
   $1\otimes\mathbb{F}_{p}(\zeta)$.
 \end{lm}


\section{The embedding evaluation problem}
\label{sec:evaluation}

